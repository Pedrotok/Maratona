%% Guilherme N. Ramos (gnramos@unb.br)

\NomeDoProblema{Fizz Buzz}%
\Conceitos{Ad-Hoc}%
\Dificuldade{1}%
\LimiteDeTempo{1}%

É uma brincadeira de números e palavras. Os jogadores jogam em turnos
incrementais, onde cada um diz um número substituindo os números divisíveis
por 3 pela palavra ``fizz'', e números divisíveis por 5 pela palavra
``buzz''. Se o número for múltiplo de 3 \emph{e} 5, diga ``fizzbuzz''.

Por exemplo:
``\emph{1, 2, fizz, 4, buzz, fizz, 7, 8, fizz, buzz, 11, fizz, 13, 14, fizzbuzz,
16, 17, fizz, 19, buzz, fizz, 22, 23, fizz, buzz, 26, fizz, 28, 29, fizzbuzz,
...}''

Faça um programa que leia um número inteiro não negativo e defina o que deve ser
dito.

\Entrada%
A entrada consiste de um número inteiro $0 < n \leq 2^{32}-1$ que deve ser
avaliado.

\Saida%
A saída deve conter o que deve ser dito conforme a entrada (uma saída por linha).

\Exemplos{8,15,21}%