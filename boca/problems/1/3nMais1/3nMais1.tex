%% Guilherme N. Ramos (gnramos@unb.br)
%%
%% Baseado em % http://uva.onlinejudge.org/index.php?option=com_onlinejudge&Itemid=8&page=show_problem&problem=36

\NomeDoProblema{$3n + 1$}% O nome completo do problema.
\Conceitos{}% Os conceitos/algoritmos abordados neste problema (ex: string, counting sort)
\LimiteDeTempo{1}%

Considere o seguinte algoritmo para gerar uma sequência de números. Comece com um
valor inteiro $n$. Se $n$ for par, divida-o por 2. Se $n$ for ímpar, multiplique-o
por 3 e some 1. Repita este processo para o novo valor de $n$, terminando quando
$n = 1$. Por exemplo, a seguinte sequência é gerada para $n = 22$:
\begin{center}22 11 34 17 52 26 13 40 20 10 5 16 8 4 2 1\end{center}

Conjectura-se que este algoritmo terminará ($n = 1$) para qualquer valor inteiro
$n$. Isto ainda não foi provado matematicamente, mas sabe-se que é verdade para
$n \in (0, 10^6)$. Para um valor $n$ de entrada, o \emph{comprimento do ciclo}
de $n$ é a quantidade de números gerados até $1$ (inclusive). Para o exemplo, o
comprimento do ciclo de 22 é 16.

Dados dois números $i$ e $j$ quaisquer, você deve construir um algoritmo que
determine o máximo ciclo para qualquer número neste intervalo. Por exemplo,
sabe-se que o maior ciclo no intervalo \mbox{[1, 10]} é 20.

\Entrada%
A entrada é formada por um par de inteiros $i$ e $j$ por linha, tais que $0< i,j < 10^6$.

\Saida%
Para o par de inteiros $i$ e $j$ fornecido, você deve imprimir $i$ e $j$, e o
máximo ciclo computado para todos os inteiros $k$ ($i \leq k \leq j$). Os inteiros
$i$ e $j$ devem aparecer na mesma ordem em que foram lidos. Estes três números
devem ser separados por espaços, todos na mesma linha, com uma linha de saída
para cada linha de entrada.

\Exemplos{21,22,23,24}%
