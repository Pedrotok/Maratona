%% http://maratona.ime.usp.br/hist/2013/primeira-fase/
\NomeDoProblema{Guarda costeira}%
\Conceitos{Ad-hoc,geometria}%
\Dificuldade{1}%

``\emph{Pega ladrão! Pega ladrão!}'' Roubaram a bolsa de uma inocente senhora que caminhava na praia da Nlogônia e o ladrão fugiu em direção ao mar. Seu plano parece óbvio: ele pretende pegar um barco e escapar!

O fugitivo, que a essa altura já está a bordo de sua embarcação de fuga, pretende seguir perpendicularmente à costa em direção ao limite de águas internacionais, que fica a 12 milhas náuticas de distância, onde estará são e salvo das autoridades locais. Seu barco consegue percorrer essa distância a uma velocidade constante de $V_F$ nós.

A Guarda Costeira pretende interceptá-lo, e sua embarcação tem uma velocidade constante de $V_G$ nós. Supondo que ambas as embarcações partam da costa exatamente no mesmo instante, com uma distância de $D$ milhas náuticas entre elas, será possível Guarda Costeira alcançar o ladrão antes do limite de águas internacionais?

Assuma que a costa da Nlogônia é perfeitamente retilínea e o mar bastante calmo, de forma a permitir uma trajetória tão retilínea quanto a costa.

\Entrada%
A entrada consiste de apenas uma linha, contendo três inteiros, $D$, $V_F$ e $V_G$, indicando respectivamente a distância inicial entre o fugitivo e a Guarda Costeira, a velocidade da embarcação do fugitivo e a velocidade da embarcação da Guarda Costeira.

\Saida%
Seu programa deve produzir uma única linha, contendo ‘S’ se for possível que a Guarda Costeira alcance o fugitivo antes que ele ultrapasse o limite de águas internacionais ou `N' caso contrário.

\paragraph{Restrições}%
\begin{itemize}
	\item $1 \leq D \leq 100$
	\item $1 \leq V_F \leq 100$
	\item $1 \leq V_G \leq 100$
\end{itemize}

\Exemplos{B_1,B_2,B_3,B_4,B_5}%
