%% Guilherme N. Ramos (gnramos@unb.br)
%%
%% UVA - 706
%% https://uva.onlinejudge.org/index.php?option=com_onlinejudge&Itemid=8&page=show_problem&problem=647

\NomeDoProblema{LCD}%
\Conceitos{string}%
\Dificuldade{2}%
\LimiteDeTempo{1}%

Matu é um engenheiro que gosta de coisas ``retrô'', e não está satisfeito com a fonte e a alta resolução de seu novo telefone quando tem de ligar para alguém. Ele pediu que você fizesse um aplicativo para adaptar o aparelho de modo a apresentar os dígitos como sua velha calculadora científica (uma HP 48 GX). Como ele vive perdendo os óculos, também quer escolher o tamanho do número...

\Entrada%
A entrada contém dois números, o valor $1 \leq t \leq 10$, indicando a escala em que mostrar o número de telefone, seguido do próprio número $0 \leq n < 1,000,000,000$ a ser discado.

\Saida%
A saída é composta em uma série de linhas, representando o número no estilo \emph{calculadora-científica-do-Matu}. Cada dígito é composto por traços verticais `\texttt{|}' e horizontais `\texttt{-}', e ocupa exatamente $t + 3$ linhas e $t + 2$ colunas de caracteres. Não se esqueça de preencher os espaços vazios com o caractere de espaço `\texttt{ }'. Há uma coluna de espaços separando números consecutivos.

\Exemplos{11,12}%
