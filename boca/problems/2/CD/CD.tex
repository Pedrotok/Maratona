%% Guilherme N. Ramos (gnramos@unb.br)

\NomeDoProblema{CD}%
\Conceitos{combinação}%
\Dificuldade{2}%
\LimiteDeTempo{1}%

Você tem uma longa jornada de carro pela frente. Você tem um gravador, mas infelizmente suas melhores músicas estão em CDs. Você precisa tê-las em fitas, então o problema a solucionar é o seguinte: você tem uma fita de $N$ minutos de duração. Como escolher faixas do CD de forma a ter o maior número de faixas na fita, mas com o menor espaço não utilizado possível?

Detalhes:
\begin{itemize}
	\item O número de faixas no CD não é maior que 20
	\item Nenhuma faixa é mais longa que $N$ minutos
	\item Faixas não repetem
	\item O comprimento de cada faixa é expressado como um número inteiro
	\item $N$ também é um inteiro
\end{itemize}

O programa também deve encontrar o conjunto de faixas que melhor preenchem a fita e deve imprimi-los na tela na mesma ordem em que estão no CD.

\Entrada%
A entrada consiste em 3 linhas, a primeira com o valor $0 < N \leq 1000$, seguido pelo número $F$ de faixas no CD e, por fim, uma linha com a duração de cada uma das faixas do CD.

\Saida%
Para cada caso de teste, imprima em uma linha o conjunto de faixas (duração em
minutos de cada faixa), e em outra linha o texto ``total:'' seguido da soma das durações.

\Exemplos{8,32,33}%
