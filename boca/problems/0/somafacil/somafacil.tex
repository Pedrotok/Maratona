%% Guilherme N. Ramos (gnramos@unb.br)

\NomeDoProblema{Mim quer somar...}% O nome completo do problema.
\Conceitos{Ad-Hoc,iniciante}% Os conceitos/algoritmos abordados neste problema (ex: string, counting sort)
\LimiteDeTempo{1}% O limite de tempo para execução da solução.

Mim gosta somar dinheiros. Portanto, leia 2 valores inteiros e os armazene nas variáveis
\texttt{salario} e \texttt{bonus}. Efetue a soma de \texttt{salario} e \texttt{bonus} atribuindo o seu
resultado a variável \texttt{renda}. Então imprima o valor de \texttt{renda} conforme exemplo
abaixo. Não apresente mensagem alguma além do especificado, e não esqueça de
imprimir o fim de linha (o famoso `\texttt{\textbackslash{n}}') após o resultado, ou
receberá o aviso de ``\emph{Presentation Error}''. Lembre-se também de retornar $0$!

\Entrada%
A entrada contém 2 valores inteiros.

\Saida%
Imprima a variável \texttt{renda} conforme exemplo abaixo, com um espaço em branco antes
e depois da igualdade.

\Exemplos{12,22}%
