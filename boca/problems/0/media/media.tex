%% Guilherme N. Ramos (gnramos@unb.br)
%%
%% Baseado em https://www.urionlinejudge.com.br/judge/pt/problems/view/1005

\NomeDoProblema{Média Fácil}%
\Conceitos{Ad-Hoc}%
\Dificuldade{0}%
\LimiteDeTempo{1}%

Leia 2 valores de ponto flutuante de dupla precisão $A$ e $B$, que correspondem a 2 notas de um aluno. A seguir, calcule a média do aluno, sabendo que a nota $A$ tem peso 3,2 e a nota $B$ tem peso 7,8 (a soma dos pesos portanto é 11).

\Entrada%
O arquivo de entrada contém $0 \leq A, B \leq 10$ valores, cada um com uma casa decimal.

\Saida%
Calcule e imprima a variável \textbf{media} conforme exemplo abaixo, com 4 dígitos após o ponto decimal e com um espaço em branco antes e depois dos caracteres `->'. Utilize variáveis de dupla precisão (double) e como todos os problemas, não esqueça de imprimir o fim de linha após o resultado, caso contrário, você receberá ``Presentation Error''.

\Exemplos{1,5,25}%
